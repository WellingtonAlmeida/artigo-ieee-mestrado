\documentclass{ieeeaccess}
\usepackage{cite}
\usepackage{amsmath,amssymb,amsfonts}
\usepackage{algorithmic}
\usepackage{graphicx}
\usepackage{textcomp}

\usepackage{bm}
\makeatletter
\AtBeginDocument{\DeclareMathVersion{bold}
\SetSymbolFont{operators}{bold}{T1}{times}{b}{n}
\SetSymbolFont{NewLetters}{bold}{T1}{times}{b}{it}
\SetMathAlphabet{\mathrm}{bold}{T1}{times}{b}{n}
\SetMathAlphabet{\mathit}{bold}{T1}{times}{b}{it}
\SetMathAlphabet{\mathbf}{bold}{T1}{times}{b}{n}
\SetMathAlphabet{\mathtt}{bold}{OT1}{pcr}{b}{n}
\SetSymbolFont{symbols}{bold}{OMS}{cmsy}{b}{n}
\renewcommand\boldmath{\@nomath\boldmath\mathversion{bold}}}
\makeatother

\def\BibTeX{{\rm B\kern-.05em{\sc i\kern-.025em b}\kern-.08em
    T\kern-.1667em\lower.7ex\hbox{E}\kern-.125emX}}

%Your document starts from here ___________________________________________________
\begin{document}
\history{Date of publication xxxx 00, 0000, date of current version xxxx 00, 0000.}
\doi{10.1109/ACCESS.2024.0429000}

\title{MPDM: A Minimalist and Progressive Documentation Model for Knowledge Management in Legacy Systems}

\author{\uppercase{Wellington Guimarães de Almeida}\authorrefmark{1,2},\uppercase{José Augusto Fabri}\authorrefmark{1}}

\address[1]{Graduate Program in Informatics (PPGI), Federal University of Technology -- Paraná (UTFPR), Cornélio Procópio, PR 86300-000, Brazil (e-mail: fabri@utfpr.edu.br)}
\address[2]{Regional Labor Court of the 3rd Region (TRT-MG), Belo Horizonte, MG 30170-910, Brazil (e-mail: walmeida@trt3.jus.br)}

\corresp{Corresponding author: Wellington G. Almeida (e-mail: walmeida@trt3.jus.br)}

\begin{abstract}
Legacy systems are essential for organizational operations but often suffer from a lack of updated technical documentation, leading to significant institutional knowledge loss. Traditional documentation methods are frequently perceived as bureaucratic and disconnected from the software development lifecycle. This paper proposes the Minimalist Progressive Documentation Model (MPDM), a framework designed to mitigate knowledge loss through a ``documentation-as-code'' approach. The MPDM is built upon two core pillars: minimalism, focusing only on essential technical and business rationale, and technological agnosticism, utilizing open formats and version control systems to ensure long-term sustainability. The model integrates into the maintenance workflow using the ``Sidecar'' and ``Strangler Fig'' patterns to incrementally modernize documentation. A qualitative case study was conducted within a Brazilian public judicial institution to evaluate the model’s feasibility. Results indicate that MPDM successfully reduced friction in knowledge capture and was organically adopted by the development team. The study demonstrates that integrating documentation into the daily developer routine through minimalist and tool-agnostic practices is an effective strategy for preserving knowledge in complex, long-lived software environments.

\end{abstract}

\begin{keywords}
Documentation-as-code, Knowledge management, Legacy systems, Minimalism, Progressive documentation, Software documentation, Software maintenance, Technological agnosticism.
\end{keywords}

\titlepgskip=-21pt

\maketitle

\section{Introduction}
\label{sec:introduction}

\PARstart{T}{he} sustainability of public institutions, such as the Regional Labor Court of the 3rd Region (TRT3) in Brazil, is heavily dependent on the continuous operation of legacy software systems. These systems often handle critical administrative and judicial processes. However, a significant challenge arises from the progressive loss of institutional knowledge. As original developers retire or leave the organization, the technical rationale behind these complex systems vanishes, primarily due to the absence of adequate and updated documentation \cite{sommerville}.

In most software maintenance environments, documentation is perceived as a bureaucratic and time-consuming task. Traditional documentation methods often fail because they are disconnected from the actual development and maintenance lifecycle. Consequently, existing records are frequently obsolete, incomplete, or stored in proprietary formats that lead to technological lock-in. This scenario creates a significant ``friction'' for developers, who must spend excessive time performing ``software archaeology'' to understand legacy code before implementing even minor changes \cite{garousi}.

To address these challenges, this study proposes the Minimalist Progressive Documentation Model (MPDM). The model is designed to be a pragmatic solution that integrates knowledge management directly into the software maintenance routine. The MPDM is based on two fundamental pillars: minimalism and technological agnosticism. Minimalism ensures that only essential information—such as architectural decisions and complex business rules—is recorded, reducing the burden on developers. Technological agnosticism, achieved through the use of open formats like Markdown and version control systems like Git, ensures that documentation remains accessible and sustainable over the long term, regardless of specific proprietary tools.

The MPDM follows an incremental approach, where knowledge is captured and recorded at the moment of maintenance, when the context is still fresh in the developer's mind. This study presents the formalization of the MPDM and its evaluation through a qualitative case study conducted with a development team at TRT3. The results demonstrate that the model promotes organic adoption and effectively preserves critical technical knowledge.

The remainder of this paper is organized as follows. Section II reviews the related work and the results of a Systematic Literature Mapping. Section III describes the proposed MPDM in detail, including its principles and workflow. Section IV presents the case study and evaluation methodology. Section V discusses the results and the perceived value of the model. Finally, Section VI concludes the paper and suggests future research directions.


\section{Related works}
The management of technical knowledge in legacy systems is a well-established challenge in software engineering. To identify the state of the art and existing gaps, a Systematic Literature Mapping (SLM) was conducted following the guidelines proposed by Kitchenham \cite{kitchenham}. The search was performed across three major databases (ACM Digital Library, IEEE Xplore, and Scopus), initially identifying 311 studies. 

The selection process was governed by the objective filters detailed in Table \ref{table_criteria}. After the removal of 85 duplicates, the remaining 226 unique studies underwent a title and abstract screening, which excluded 179 papers. The remaining 47 studies were subjected to a full-text quality assessment, resulting in the final selection of 22 primary studies, plus one additional study suggested during the qualification process, totaling 23 papers for detailed analysis. The distribution of these studies across the analyzed research metrics and the identified research gaps are summarized in Table \ref{tab:slm_stats}.

\begin{table}
\caption{\textbf{Inclusion and Exclusion Criteria (IC/EC)}}
\label{table_criteria}
\setlength{\tabcolsep}{3pt}
\begin{tabular}{|p{25pt}|p{206pt}|}
\hline
ID & Criterion Description \\
\hline
\multicolumn{2}{|c|}{\textit{Inclusion Criteria}} \\
\hline
IC1 & Studies addressing knowledge management tools or practices in software development contexts. \\
IC2 & Research discussing minimalist or technologically agnostic approaches to software documentation. \\
IC3 & Peer-reviewed papers published in recognized computer science journals or conferences. \\
IC4 & Studies covering activities related to the software lifecycle, such as maintenance and versioning. \\
\hline
\multicolumn{2}{|c|}{\textit{Exclusion Criteria}} \\
\hline
EC1 & Studies not directly related to software engineering or knowledge management. \\
EC2 & Works focused exclusively on proprietary tools without considering agnostic aspects. \\
EC3 & Non-academic sources, gray literature, or papers that were not peer-reviewed. \\
EC4 & Duplicate documents or studies with insufficient relevance after title and abstract analysis. \\
\hline
\end{tabular}
\end{table}

The SLM revealed that while 56\% of the studies (13 papers) address the documentation of legacy systems directly \cite{garousi}, and another 13 papers propose structured methodologies or frameworks, there is a clear trend toward high-complexity technical solutions. For instance, several studies focus on automated reverse engineering, such as the use of PageRank algorithms to identify key classes \cite{e02} or Latent Dirichlet Allocation (LDA) to recover architectural layers \cite{e03}. While these automated approaches are valuable for technical recovery, they often overlook the human factor and the continuous ``friction'' that developers face during daily maintenance.

A critical gap was identified regarding the simplicity and adoption of these models. Only 30\% of the analyzed studies (7 papers) explicitly emphasized minimalism or the reduction of bureaucracy as a core pillar. Furthermore, the focus on employee engagement—essential for the sustainability of any knowledge management initiative—was even scarcer, addressed by only 26\% of the studies (6 papers).

Moreover, technological agnosticism, a key requirement to avoid proprietary lock-in in public institutions, is rarely combined with minimalist practices. Most existing tools require specific platforms or complex setups that do not align with the agile, low-friction needs of maintenance teams. Finally, only 13\% of the literature (3 studies) discussed these challenges within the specific context of public sector or judicial institutions, where resource constraints and high staff turnover are prevalent.

The MPDM proposed in this paper distinguishes itself by addressing this specific void. Unlike heavy frameworks or purely automated tools, MPDM focuses on a "documentation-as-code" philosophy that balances technical rigor with process lightness, ensuring that knowledge capture becomes a natural extension of the developer’s workflow rather than an additional bureaucratic burden.

\begin{table}
\caption{\textbf{SLM Metrics and Study Distribution (N=23)}}
\label{tab:slm_stats}
\setlength{\tabcolsep}{3pt}
\begin{tabular}{|p{25pt}|p{145pt}|p{20pt}|p{20pt}|}
\hline
ID & Metric Description & Count & \% \\
\hline
M1 & Focus on legacy systems documentation & 13 & 56.5\% \\
M2 & Methodology or framework proposal & 13 & 56.5\% \\
M3 & \textbf{Emphasis on simplicity and ease of adoption} & \textbf{7} & \textbf{30.4\%} \\
M4 & Tool-agnostic approach & 11 & 47.8\% \\
M5 & \textbf{Focus on employee engagement} & \textbf{6} & \textbf{26.1\%} \\
M6 & Practical validation in real scenarios & 18 & 78.3\% \\
M7 & \textbf{Public or judicial sector context} & \textbf{3} & \textbf{13.0\%} \\
M8 & Documentation quality/efficiency improvement & 16 & 69.6\% \\
\hline
\multicolumn{4}{p{240pt}}{The low percentages in M3, M5, and M7 (highlighted) indicate the research gap addressed by the proposed MPDM.}
\end{tabular}
\end{table}

\section{The Minimalist Progressive Documentation Model (MPDM)}
The Minimalist Progressive Documentation Model (MPDM) is proposed as a pragmatic alternative to traditional, bureaucratic documentation processes. It focuses on reducing friction and ensuring that technical knowledge is captured incrementally during the software maintenance lifecycle. The model is built upon four fundamental principles and a lightweight operational workflow.

\subsection{Core Principles}

\subsubsection{Minimalism and Focus on the Essential}
Documentation should avoid describing the obvious. Instead, MPDM focuses on capturing technical and business rationales that are not evident in the source code, such as complex business rules, architectural decisions, and critical configuration steps \cite{garousi}.

\subsubsection{Technological Agnosticism and Longevity}
To prevent institutional knowledge from being trapped in proprietary tools, MPDM mandates the use of open, durable formats. Plain text files (Markdown) and version control systems (Git) are used to ensure that documentation remains accessible regardless of future tool migrations \cite{jesson}.

\subsubsection{Incremental and Contextual Nature}
Documentation is not a separate phase at the beginning or end of development. It is a continuous process integrated into the daily maintenance routine. Knowledge is recorded when the context is fresh in the developer's mind, improving accuracy and reducing effort.

\subsubsection{Collective Responsibility}
Knowledge management is a shared duty. Practices such as peer reviews (Merge Requests) are applied to documentation files, ensuring that the entire team contributes to the clarity and accuracy of the institutional base \cite{nonaka}.

\subsection{Artifacts and structure}

The central artifact of MPDM is the Knowledge Base, which is implemented as a Git repository. This approach provides native versioning, authorship history, and traceability. Documentation follows the ``Sidecar'' pattern \cite{microsoft}, where files reside alongside or are directly linked to the source code they describe.

Each knowledge unit is a Markdown file containing:

\begin{itemize}
\item Context/Problem: A brief explanation of why the documentation is necessary.
\item Key Points: Technical details, business logic, or architectural motives.
\item Code References: Direct links to specific files, classes, or functions in the repository.
\item Visual Examples: Simple diagrams generated from text (e.g., Mermaid.js) to explain complex flows without the overhead of binary image files.
\end{itemize}

\subsection{operational workflow}

The MPDM workflow is triggered by real maintenance demands rather than a predefined schedule. It consists of four stages:

\begin{enumerate}
    \item Trigger: A maintenance task (bug fix or new feature) is assigned to a developer.
    \item Investigation: The developer explores the legacy code to understand its current behavior, effectively converting explicit code into tacit knowledge.
    \item Minimal Registration: During or immediately after the task, the developer performs a "minimalist" record of the non-obvious knowledge acquired.
    \item Consolidation: The documentation change is included in the same commit or Merge Request as the code change. Peer review validates both the logic and the associated documentation before it is integrated into the main repository.
\end{enumerate}

The MPDM serves as a conceptual framework that requires instantiation according to the organizational context. The following section details its instantiation within a public judicial environment.

\section{Case study and evaluation}
To evaluate the feasibility and perceived value of the MPDM, a qualitative case study was conducted within a development team at the Regional Labor Court of the 3rd Region (TRT3). The study aimed to determine if a minimalist, agnostic, and incremental approach could effectively mitigate technical knowledge loss in a public judicial institution.

\subsection{Scenarion and team characterization}
The study involved a team of six professionals: five developers and one section head. The team's seniority was diverse, comprising three senior, one mid-level, and one junior developer.

A critical factor in this scenario was the heterogeneous distribution of technical knowledge regarding legacy technologies. Only two members possessed intermediate knowledge of the primary legacy technology (ZIM/Oracle), while three members had no prior contact with it. This concentration of knowledge in a few individuals represented a high institutional risk, making the team an ideal environment for testing the MPDM. The primary object of study was a critical judicial application developed using ZIM/Oracle, which suffered from a severe lack of formal documentation.

\subsection{Planning and infrastructure}
The implementation was designed to minimize friction and leverage existing tools. The following steps were executed:
\begin{enumerate}
    \item Central Repository: A dedicated Git repository was established on the institution's GitLab server to serve as the centralized Knowledge Base. Git was chosen because it was already the standard tool for source code versioning, eliminating the learning curve.
    \item Initial Structure and Metadata: To ensure replicability, a standardized directory structure was adopted, organized by project and module (e.g., /project-name/module-a/). Each Markdown file was required to follow a minimalist header containing a descriptive title, the context of the maintenance task, and the rationale for the technical decision. This setup prevents information fragmentation and guides developers in another environment to follow the same logic.
    \item Alignment Session: The MPDM principles and workflow were formally presented to the team. The focus was on explaining the ``documentation-as-code'' philosophy and the incremental nature of the model.
\end{enumerate}

\subsection{Execution and validation process}
Once the infrastructure was ready, the MPDM workflow was integrated into the team's daily routine. Documentation was triggered by real maintenance demands (bug fixes or minor evolutions).

To ensure quality and consistency, a validation cycle was established. Similar to a standard code review process, the insertion or update of any documentation was conditioned on the approval of Merge Requests. Senior members with greater technical and business expertise were appointed as reviewers, materializing the principle of collective responsibility and ensuring that the registered knowledge was both accurate and useful \cite{antoniol}.

The evaluation was conducted over a six-month period, utilizing direct observation and periodic evaluation of the generated knowledge. This approach allows the capture of nuanced perceptions regarding ``friction'' and ``perceived value,'' which are critical for evaluating minimalist models where traditional objective metrics, such as code-to-comment ratios or line counts, may fail to represent the actual quality and utility of the documentation produced.

\section{Results and discussion}
The evaluation of the MPDM revealed a combination of immediate benefits and contextual challenges that provide significant insights into technical knowledge management in public institutions.

\subsection{Observed adoption and benefits}
The acceptance of the proposed tools (Git, GitLab, and Markdown) was unanimous and immediate. Since these tools were already part of the developers' coding routine, the ``friction'' typically associated with learning new documentation platforms was non-existent.

A major result was the organic expansion of the model. Although the study initially focused on legacy systems, the team proactively suggested applying MPDM to all new projects. This led to a hybrid architecture where the central repository acted as a portal, consolidating links to 32 existing projects and 9 new proofs of concept. This transition indicates that developers perceive minimalist documentation as a high-value investment when integrated into the development lifecycle.

\subsection{The challenge of legacy friction}
While adoption was successful for new systems, documentation for legacy code faced higher resistance. This does not indicate a failure of the MPDM, but rather reflects the reality of maintenance environments. Legacy systems often involve high cognitive load and urgent, reactive demands, which compete for the developer's time. In this context, documenting legacy code was perceived as a ``high-friction'' task compared to documenting new features. This finding suggests that for legacy environments, the ``trigger'' for documentation must be strictly aligned with the most critical and frequently maintained modules.

\subsection{Strategic metaphors: Sidecar and Strangler Fig}
The success of the MPDM can be explained through two architectural metaphors:

\subsubsection{Sidecar Pattern}
By keeping documentation in the same repository and lifecycle as the source code, the MPDM operates as a ``sidecar'' component \cite{microsoft}. It extends the value of the code without altering its core functionality, ensuring that both evolve in synchrony.

\subsubsection{Strangler Fig Pattern}
The MPDM adopts a philosophy of incremental replacement rather than massive migration. The new Knowledge Base acts as a modern ``facade'' that intercepts information searches. Over time, as new documentation is added during maintenance, the new base ``strangles'' the old, obsolete documentation in an organic and safe manner \cite{fowler}.
\subsection{Qualitative validation}
The study confirmed that a minimalist and agnostic approach fosters institutional longevity. By avoiding proprietary formats, the institution ensures that technical knowledge remains a long-term asset, immune to vendor lock-in. Furthermore, the use of Git for documentation provided unprecedented transparency and traceability, allowing the team to understand not only what changed in the business logic, but who changed it and why.

\section{Conclusion}
Technical knowledge loss in legacy systems represents a critical risk to the sustainability of public and private institutions. This study proposed the Minimalist Progressive Documentation Model (MPDM) to mitigate this risk through a pragmatic, ``documentation-as-code'' approach. By integrating documentation directly into the software maintenance lifecycle using open formats and version control, the MPDM successfully reduced the friction typically associated with traditional, bureaucratic documentation methods.

The primary contribution of this research is the systematization of minimalist and agnostic principles into a cohesive framework. The application of the ``Sidecar'' and ``Strangler Fig'' patterns to knowledge management provides a novel strategy for incrementally modernizing technical records without the need for disruptive migrations. The case study conducted at the Regional Labor Court (TRT3) demonstrated that the model is highly compatible with modern development routines, leading to organic adoption by the team and the consolidation of knowledge across dozens of projects.

Despite these contributions, this study has limitations. The case study was conducted within a single organizational context, and the low volume of maintenance demands for legacy systems during the study period limited the generation of a larger volume of new documentation. Furthermore, while the evaluation relied primarily on qualitative data (perceptions and observations) without long-term objective metrics, such as precise onboarding time reduction, this approach was chosen to capture the cultural shift and team engagement, which are the primary barriers to documentation sustainability. Future works should address this by implementing quantitative KPIs to complement the findings.

Future research should focus on validating the MPDM in diverse organizational settings, including private sector companies with different maturity levels. Additionally, there is significant potential for automation, such as the development of tools to generate navigable web portals from Markdown repositories and the creation of ``linters'' or automated validators to ensure documentation consistency. Finally, developing quantitative metrics to measure the impact of MPDM on developer onboarding time and maintenance efficiency would provide robust empirical evidence of its long-term benefits.

\appendices
\section{\break Summary of Selected Primary Studies}
Table \ref{tab:slm_summary} presents a synthesized overview of the 23 primary studies identified through the Systematic Literature Mapping (SLM). These studies provided the basis for identifying the research gaps in legacy system documentation that the MPDM aims to address.

\begin{table*}[h]
\caption{Overview of Primary Studies from the Systematic Mapping}
\label{tab:slm_summary}
\centering
\begin{tabular}{|c|p{150pt}|p{300pt}|}
\hline
\textbf{ID} & \textbf{Reference} & \textbf{Core Contribution / Scope} \\
\hline
E01 & Fehlmann and Falkner (2015) \cite{e01} & Investigates Agile (Scrum) adaptation for the continuous evolution of long-lived legacy software systems. \\ \hline
E02 & Chirila and Şora (2019) \cite{e02} & Proposes optimization of classifiers based on PageRank to identify key classes in object-oriented systems with poor documentation. \\ \hline
E03 & Belle et al. (2016) \cite{e03} & Recovers software architecture layers by combining lexical analysis (LDA) and structural source code information. \\ \hline
E04 & Nallusamy and Zulkifle (2021) \cite{e04} & Evaluates an ontology-based approach (OBSR) for software redocumentation to support program comprehension. \\ \hline
E05 & Hadge and Shenoy (2024) \cite{e05} & Explores Machine Learning techniques for drift detection in legacy systems as requirements and environments evolve. \\ \hline
E06 & Moser and Pichler (2021) \cite{e06} & Presents the "eknows" platform for multi-language reverse engineering and automated documentation generation. \\ \hline
E07 & Alanazi et al. (2021) \cite{e07} & Facilitates program comprehension through multi-level hierarchical abstractions of static call graphs. \\ \hline
E08 & Kamran et al. (2017) \cite{e08} & Proposes a dynamic analysis heuristic to suggest key classes for initial program investigation, reducing comprehension time. \\ \hline
E09 & Ramadani and Wagner (2017) \cite{e09} & Investigates how coupled file changes in version control influence developer help-seeking behavior during maintenance. \\ \hline
E10 & Gerdes et al. (2018) \cite{e10} & Identifies technology features to assist architects in understanding and maintaining legacy software architectures. \\ \hline
E11 & Hu et al. (2015) \cite{e15} & Models the evolution of development topics using Dynamic Topic Models (DTM) to capture changes in software focus over time. \\ \hline
E12 & Scalabrino (2017) \cite{e12} & Recommends entry-point code artifacts using knowledge graphs to reduce comprehension effort for new developers. \\ \hline
E13 & Link et al. (2019) \cite{e13} & Proposes a responsibility-oriented architecture recovery method based on text classification. \\ \hline
E14 & Antoniol et al. (2025) \cite{antoniol} & Provides a retrospective on using Information Retrieval to recover traceability links between source code and free-text documentation. \\ \hline
E15 & Kadar et al. (2015) \cite{e15} & Uses ontologies and UML modeling to generate semantic summaries of source code for maintenance teams. \\ \hline
E16 & Leemans et al. (2018) \cite{e16} & Applies process mining to event logs to analyze and understand the behavior of complex legacy systems. \\ \hline
E17 & Alzahrani (2024) \cite{e17} & A systematic review of software documentation trends, highlighting the lack of prospective documentation models. \\ \hline
E18 & Al-Saiyd (2017) \cite{e18} & Proposes a bottom-up static comprehension strategy to improve code readability and maintainer productivity. \\ \hline
E19 & Terry and Chandrasekar (2025) \cite{e19} & Identifies engineering barriers to legacy system evolution and proposes formal assessment methods. \\ \hline
E20 & Javed et al. (2016) \cite{e20} & Presents a pattern language for constructing and maintaining traceability links between architecture, requirements, and code. \\ \hline
E21 & Garousi et al. (2015) \cite{garousi} & Industrial case study confirming that updated and concise documentation has the highest impact on technical utility. \\ \hline
E22 & Balde et al. (2024) \cite{e22} & Develops dynamic visualization tools to generate UML sequence and class diagrams for legacy system comprehension. \\ \hline
E23 & L’Erario et al. (2020) \cite{e23} & Proposes a software maintenance process model for SMEs, focusing on knowledge recovery and turnover impact. \\ \hline
\end{tabular}
\end{table*}


\begin{thebibliography}{00}

\bibitem{sommerville} I. Sommerville, \emph{Software Engineering}, 9th ed. São Paulo, Brazil: Pearson, 2011.

\bibitem{garousi} V. Garousi, G. Garousi-Yusifoglu, V. Ruhe, G. Zhi, J. Moussavi, and B. Smith, ``Usage and usefulness of technical software documentation: An industrial case study,'' \emph{Information and Software Technology}, vol. 57, pp. 664--682, 2015.

\bibitem{kitchenham} B. Kitchenham and S. Charters, ``Guidelines for performing Systematic Literature Reviews in Software Engineering,'' Keele University, UK, Tech. Rep., 2007.

\bibitem{e01} S. Fehlmann and K. Falkner, ``A case study in agility and evolving the long-lived software system,'' in \emph{Proc. ASWEC 2015 24th Australasian Software Engineering Conference}, 2015, pp. 33--37.

\bibitem{e02} C.-B. Chirila and I. Şora, ``An analysis on the optimization of the weights scheme for a rule based key class classifier in object-oriented systems,'' in \emph{Proc. 23rd Int. Conf. on System Theory, Control and Computing (ICSTCC)}, 2019, pp. 413--418.

\bibitem{e03} A. B. Belle, G. E. Boussaidi, and S. Kpodjedo, ``Combining lexical and structural information to reconstruct software layers,'' \emph{Information and Software Technology}, vol. 74, pp. 1--16, 2016.

\bibitem{e04} M. H. H. S. Nallusamy and F. A. Zulkifle, ``Controlled experiment for assessing the contribution of ontology based software redocumentation approach to support program understanding,'' \emph{Computing and Informatics}, vol. 40, no. 5, pp. 1025--1055, 2021.

\bibitem{e05} S. Hadge and G. S. Shenoy, ``Drift detection in legacy systems using machine learning techniques,'' in \emph{Proc. 2024 3rd Int. Conf. for Innovation in Technology (INOCON)}, 2024, pp. 1--6.

\bibitem{e06} M. Moser and J. Pichler, ``eknows: Platform for multi-language reverse engineering and documentation generation,'' in \emph{Proc. IEEE 2021 Int. Conf. on Software Maintenance and Evolution (ICSME)}, 2021, pp. 559--568.

\bibitem{e07} R. Alanazi, G. Gharibi, and Y. Lee, ``Facilitating program comprehension with call graph multilevel hierarchical abstractions,'' \emph{Journal of Systems and Software}, vol. 176, p. 110945, 2021.

\bibitem{e08} M. Kamran, M. Ali, and A. Ahmed, ``Generating suggestions for initial program investigation using dynamic analysis,'' in \emph{Proc. 2017 Int. Conf. on Communication, Computing and Digital Systems (C-CODE)}, 2017, pp. 233--237.

\bibitem{e09} J. Ramadani and S. Wagner, ``How do coupled file changes influence how developers seek help during maintenance tasks?'' in \emph{Proc. 2017 IEEE Int. Conf. on Software Quality, Reliability and Security (QRS)}, 2017, pp. 410--417.

\bibitem{e10} S. Gerdes, T. Fechner, and M. Riebisch, ``Identification of technology features to understand and maintain software architectures,'' in \emph{Proc. 2018 25th Australasian Software Engineering Conference (ASWEC)}, 2018, pp. 210--214.

\bibitem{e11} J. Hu, X. Sun, D. Lo, and B. Li, ``Modeling the evolution of development topics using dynamic topic models,'' in \emph{Proc. 2015 IEEE 22nd Int. Conf. on Software Analysis, Evolution, and reengineering (SANER)}, 2015, pp. 3--12.

\bibitem{e12} S. Scalabrino, ``On software odysseys and how to prevent them,'' in \emph{Proc. IEEE 2017 IEEE/ACM 39th Int. Conf. on Software Engineering Companion (ICSE-C)}, 2017, pp. 91--93.

\bibitem{e13} D. Link, P. Behnamghader, R. Moazeni, and B. Boehm, ``Recover and relax: concern-oriented software architecture recovery for systems development and maintenance,'' in \emph{Proc. 2019 IEEE/ACM Int. Conf. on Software and System Processes (ICSSP)}, 2019, pp. 64--73.

\bibitem{antoniol} G. Antoniol, G. Canfora, G. Casazza, A. D. Lucia, and E. Merlo, ``Recovering traceability links between code and documentation: a retrospective,'' \emph{IEEE Transactions on Software Engineering}, 2025.

\bibitem{e15} R. Kadar, S. M. Syed-Mohamad, and N. A. Rashid, ``Semantic-based extraction approach for generating source code summary towards program comprehension,'' in \emph{Proc. IEEE 2015 9th Malaysian Software Engineering Conference (MySEC)}, 2015, pp. 129--134.

\bibitem{e16} M. Leemans, W. M. van der Aalst, M. G. van den Brand, R. R. Schiffelers, and L. Lensink, ``Software process analysis methodology—a methodology based on lessons learned in embracing legacy software,'' in \emph{Proc. IEEE 2018 IEEE Int. Conf. on Software Maintenance and Evolution (ICSME)}, 2018, pp. 665--674.

\bibitem{e17} A. A. Alzahrani, ``Software systems documentation: A systematic review,'' \emph{International Journal of Advanced Computer Science \& Applications}, vol. 15, no. 8, 2024.

\bibitem{e18} N. A. Al-Saiyd, ``Source code comprehension analysis in software maintenance,'' in \emph{Proc. IEEE 2017 2nd Int. Conf. on Computer and Communication Systems (ICCCS)}, 2017, pp. 1--5.

\bibitem{e19} S. Terry and V. Chandrasekar, ``Systems engineering barriers to legacy system evolution: Legacy system assessment,'' \emph{Systems Engineering}, vol. 28, no. 2, pp. 207--223, 2025.

\bibitem{e20} M. A. Javed, S. Stevanetic, and U. Zdun, ``Towards a pattern language for construction and maintenance of software architecture traceability links,'' in \emph{Proc. 21st European Conf. on Pattern Languages of Programs}, 2016, pp. 1--20.

\bibitem{e22} K. Balde, L. Prasad, and R. Yadav, ``Visualizing complexity: A dynamic approach to reverse engineering,'' in \emph{Proc. IEEE 2024 Int. Conf. on Advances in Computing Research on Science Engineering and Technology (ACROSET)}, 2024, pp. 1--7.

\bibitem{e23} A. L’Erario, H. C. S. Thomazinho, and J. A. Fabri, ``An approach to software maintenance: A case study in small and medium-sized businesses IT organizations,'' \emph{International Journal of Software Engineering and Knowledge Engineering}, vol. 30, no. 05, pp. 603--630, 2020.

\bibitem{jesson} G. Jesson, \emph{Technology Agnosticism: Embracing the New Age of IT}. Cambridge, U.K.: IT Governance Ltd., 2015.

\bibitem{nonaka} I. Nonaka and H. Takeuchi, \emph{The Knowledge-Creating Company: How Japanese Companies Create the Dynamics of Innovation}. Oxford, U.K.: Oxford Univ. Press, 1995.

\bibitem{microsoft} Microsoft, ``Sidecar pattern,'' Sep. 2023. [Online]. Available: https://learn.microsoft.com/en-us/azure/architecture/patterns/sidecar

\bibitem{fowler} M. Fowler, ``StranglerFigApplication,'' Jun. 2004. [Online]. Available: https://martinfowler.com/bliki/StranglerFigApplication.html

\end{thebibliography}

\begin{IEEEbiographynophoto}{Wellington Guimarães de Almeida} 
received the B.S. degree in Electronic and Telecommunications Engineering from the Pontifical Catholic University of Minas Gerais (PUC-MG), Brazil, in 2002. He completed a post-graduate specialization in Software Engineering in 2005 and another in Geoprocessing in 2015, both at PUC-MG. He is currently pursuing an M.S. degree in Computer Science (Applied Computing) with a research focus on Software Engineering at the Federal University of Technology – Paraná (UTFPR), Brazil.

His professional career began in the telecommunications and engineering sectors, where he focused on the design and configuration of telecommunications networks. During this period, he held technical and analytical roles at major companies such as Oi and Cemig Telecom. He subsequently transitioned his career toward Information Technology, specifically within the Brazilian public sector. For the past 12 years, he has been with the Regional Labor Court of the 3rd Region (TRT-MG), serving as a Network Analyst for two years before dedicating the last decade to the role of Systems Analyst.

In his current tenure as a technical lead at TRT-MG, he has spearheaded several national-scale software projects within the Brazilian Labor Justice system. His work is characterized by a drive for innovation and the research of minimalist methodologies aimed at enhancing software development processes and institutional information management.
\end{IEEEbiographynophoto}

\begin{IEEEbiographynophoto}{José Augusto Fabri}
received the B.S. degree in Data Processing Technology from the Educational Foundation of the Municipality of Assis (FEMA), Brazil, in 1997. He received the M.S. degree in Computer Science from the Federal University of São Carlos (UFSCar), Brazil, in 1999, and the Ph.D. degree in Production Engineering from the Polytechnic School of the University of São Paulo (USP), Brazil.

He is currently an Adjunct Professor with the Federal University of Technology – Paraná (UTFPR), Brazil, where he serves as the Pro-Rector of Undergraduate and Professional Education (PROGRAD). He has extensive experience in academic leadership and the coordination of higher education policies.

His research interests include Software Engineering, Software Production Processes, and Software Factories. Dr. Fabri’s work focuses on optimizing development workflows and implementing structured methodologies in software production environments.
\end{IEEEbiographynophoto}

\EOD

\end{document}
